\documentclass[UTF8,12pt]{article} % 12pt 为字号大小
\usepackage{amssymb,amsfonts,amsmath,amsthm}
\usepackage{times}
\usepackage{graphicx} % 插图
\usepackage{cite}
\usepackage{xeCJK}
\usepackage{placeins} % 防止浮动
%----------
% 插入代码的格式定义
% 参考 https://www.latexstudio.net/archives/5900.html
%----------
\usepackage{listings}
\lstset{
 columns=fixed,       
 numbers=left,                                        % 在左侧显示行号
 numberstyle=\tiny\color{gray},                       % 设定行号格式
 frame=none,                                          % 不显示背景边框
 backgroundcolor=\color[RGB]{245,245,244},            % 设定背景颜色
 keywordstyle=\color[RGB]{40,40,255},                 % 设定关键字颜色
 numberstyle=\footnotesize\color{darkgray},           
 commentstyle=\it\color[RGB]{0,96,96},                % 设置代码注释的格式
 stringstyle=\rmfamily\slshape\color[RGB]{128,0,0},   % 设置字符串格式
 showstringspaces=false,                              % 不显示字符串中的空格
 language=c++,                                        % 设置语言
}
%----------
% 算法伪代码
% https://blog.csdn.net/lwb102063/article/details/53046265
%----------
\usepackage{algorithm}  
\usepackage{algpseudocode}  
\usepackage{amsmath}  
\renewcommand{\algorithmicrequire}{\textbf{Input:}}  % Use Input in the format of Algorithm  
\renewcommand{\algorithmicensure}{\textbf{Output:}} % Use Output in the format of Algorithm  

%----------
% 字体定义
%----------
\setCJKmainfont[
    BoldFont = {Noto Sans CJK SC Bold},
    ItalicFont = {AR PL UKai CN}
]{Noto Serif CJK SC}

\setCJKsansfont{Noto Sans CJK SC}
\setCJKfamilyfont{zhsong}{Noto Serif CJK SC}  % 宋体
\setCJKfamilyfont{zhhei}{Noto Sans CJK SC}    % 黑体
\setCJKfamilyfont{zhkai}{AR PL UKai CN}      % 楷体
\setCJKfamilyfont{zhfs}{AR PL UMing CN}      % 仿宋(明体替代)
\setCJKfamilyfont{zhli}{Noto Sans CJK SC}    % 隶书(黑体替代)
\setCJKfamilyfont{zhyou}{Noto Sans CJK SC}   % 圆体(黑体替代)

\newcommand*{\songti}{\CJKfamily{zhsong}} % 宋体
\newcommand*{\heiti}{\CJKfamily{zhhei}}   % 黑体
\newcommand*{\kaiti}{\CJKfamily{zhkai}}  % 楷体
\newcommand*{\fangsong}{\CJKfamily{zhfs}} % 仿宋
\newcommand*{\lishu}{\CJKfamily{zhli}}    % 隶书
\newcommand*{\yuanti}{\CJKfamily{zhyou}} % 圆体

%----------
% 版面设置
%----------
%首段缩进
\usepackage{indentfirst}
\setlength{\parindent}{2em}
%行距
\renewcommand{\baselinestretch}{1.25} % 1.25倍行距
%页边距
\usepackage[a4paper]{geometry}
\geometry{verbose,
  tmargin=2cm,% 上边距
  bmargin=2cm,% 下边距
  lmargin=2cm,% 左边距
  rmargin=2cm % 右边距
}

% ----------
% 多级标题格式在此设置
% https://zhuanlan.zhihu.com/p/32712209
% \titleformat{command}[shape]%定义标题类型和标题样式,字体
% {format}%定义标题格式:字号(大小),加粗,斜体
% {label}%定义标题的标签,即标题的标号等
% {sep}%定义标题和标号之间的水平距离
% {before-code}%定义标题前的内容
% [after-code]%定义标题后的内容
% ----------
\usepackage{titlesec} %自定义多级标题格式的宏包
% 三级标题
% 4
\titleformat{\section}[block]{\large \bfseries}{\arabic{section}}{1em}{}[]
% 4.1
\titleformat{\subsection}[block]{\normalsize \bfseries}{\arabic{section}.\arabic{subsection}}{1em}{}[]
% 4.1.1
\titleformat{\subsubsection}[block]{\small \mdseries}{\arabic{section}.\arabic{subsection}.\arabic{subsubsection}}{1em}{}[]
\titleformat{\paragraph}[block]{\footnotesize \bfseries}{[\arabic{paragraph}]}{1em}{}[]


%----------
% 其他宏包
%----------
%图形相关
\usepackage[x11names]{xcolor} % must before tikz, x11names defines RoyalBlue3
\usepackage{graphicx}
\usepackage{pstricks,pst-plot,pst-eps}
\usepackage{subfig}
\def\pgfsysdriver{pgfsys-dvipdfmx.def} % put before tikz
\usepackage{tikz}

%原文照排
\usepackage{verbatim}

%链接的格式
\usepackage[colorlinks,linkcolor=red]{hyperref}
%表格
\usepackage{tabularx}

%==========
% 正文部分
%==========

\begin{document}

% \kaiti 是楷体,参见上面的字体设置
\title{\bf{\kaiti 中医证型分类模型的构建}}
\author{姓名:胡屹莹\hspace{1cm}时间:2026.1.29}
\date{}
\maketitle

\abstract{
本研究基于259个病例样本数据及提取的78个特征数据,
包括舌诊(苔质,舌色等)/脉诊(平,浮,洪等)/问诊
(呕吐痰涎,脘腹痞满)等,进行中医证型分类模型的构建。
}
\paragraph{\bf{ \kaiti 关键词:中医辨证,机器学习,}}
\paragraph{\\}

\section{介绍(Introduction)}
\subsection{背景(Background)}
  心肾阳虚证

\subsection{任务(Main Task)}
在259个病例样本数据中,使用随机森林和KNN算法建模,
并在五折交叉验证后,进行优化调参;

\section{方法(Method)}

\subsection{数据分析与预处理}
在259个病例样本数据中,共有78个特征指标,其中面诊指标24个,舌诊指标25个,脉诊指标29个;分类指标为中医证型,其中包含8个类别,分别是气虚血瘀证,心肾阳虚证,痰浊闭阻证,心血瘀阻证,气滞血瘀证,气阴两虚证,心肾阴虚证,寒凝心脉证,8种证型占比均在12\%左右,数据数量分布比较均匀。

\noindent
\begin{table}[H] % [H]选项可以强制将表格放置在当前位置(需要\usepackage{float})
  \centering
  \smallskip % 在表格上方增加一点间距
  \begin{tabular}{ccc}
    \hline
    \noalign{\smallskip}
    中医证型 & 样本数量 & 百分比(\%) \\
    \noalign{\smallskip}
    \hline
    \noalign{\smallskip}
    气虚血瘀证 & 35 & 13.5 \\
    心肾阳虚证 & 34 & 13.1 \\
    痰浊闭阻证 & 34 & 13.1 \\
    心血瘀阻证 & 33 & 12.7 \\
    气滞血瘀证 & 32 & 12.4 \\
    气阴两虚证 & 31 & 12.0 \\
    心肾阴虚证 & 30 & 11.6 \\
    寒凝心脉证 & 30 & 11.6 \\
    \noalign{\smallskip}
    \hline
  \end{tabular}
  \caption{中医证型样本分布统计}
  \label{tab:syndrome-dist}
\end{table}

本研究通过卡方检验对8种冠心病中医证型与30项临床特征进行关联性分析,发现其中10项特征具有极显著的统计学差异($p_{\text{adj}} < 0.001$)。其中,“五心烦热”与“形寒肢冷”的鉴别能力最为突出,卡方值分别为174.44和154.71,Cramer's $V$值分别为0.821和0.773,表明二者与证型之间存在强关联,可作为区分阴虚证型与阳虚证型的关键指标。此外,舌象特征如“腻苔”($V=0.511$)、“涩脉”($V=0.496$)及“舌色红”($V=0.442$)亦表现出中等程度的鉴别价值,提示舌诊在证型辨别中具有重要作用。症状方面,“瘀点瘀斑”($V=0.468$)、“乏力”($V=0.440$)等特征虽效应量稍弱,但仍能提供一定的辨证参考。以上结果经Benjamini-Hochberg多重检验校正后仍保持显著,说明本研究的发现具有较好的统计可靠性。

\noindent
\begin{table}[H]
  \centering
  \smallskip
  \begin{tabular}{cccccc}
    \hline
    \noalign{\smallskip}
    排名 & 特征 & 卡方值 & $p$值 & $p_{\text{adj}}$ & Cramer's V \\
    \noalign{\smallskip}
    \hline
    \noalign{\smallskip}
    1 & 五心烦热 & 174.4441 & $2.90 \times 10^{-34}$ & $1.77 \times 10^{-32}$ & 0.8207 \\
    2 & 形寒肢冷 & 154.7093 & $4.16 \times 10^{-30}$ & $1.27 \times 10^{-28}$ & 0.7729 \\
    3 & 腻苔 & 67.5523 & $4.61 \times 10^{-12}$ & $9.37 \times 10^{-11}$ & 0.5107 \\
    4 & 涩 & 63.7917 & $2.63 \times 10^{-11}$ & $4.01 \times 10^{-10}$ & 0.4963 \\
    5 & 瘀点瘀斑 & 56.6626 & $6.98 \times 10^{-10}$ & $7.09 \times 10^{-9}$ & 0.4677 \\
    6 & 细 & 56.6827 & $6.91 \times 10^{-10}$ & $7.09 \times 10^{-9}$ & 0.4678 \\
    7 & 数 & 54.2367 & $2.11 \times 10^{-9}$ & $1.84 \times 10^{-8}$ & 0.4576 \\
    8 & 舌色红 & 50.7076 & $1.05 \times 10^{-8}$ & $8.00 \times 10^{-8}$ & 0.4425 \\
    9 & 乏力 & 50.0352 & $1.42 \times 10^{-8}$ & $9.64 \times 10^{-8}$ & 0.4395 \\
    10 & 紧 & 46.6683 & $6.48 \times 10^{-8}$ & $3.95 \times 10^{-7}$ & 0.4245 \\
    \noalign{\smallskip}
    \hline
  \end{tabular}
  \caption{各特征在8种中医证型间的卡方检验结果(前10项)}
  \label{tab:chi-square-results}
\end{table}

\subsection{随机森林和KNN模型的搭建}
本研究的调参逻辑采用分层交叉验证框架,通过基线评估确定模型潜力后,
针对二值特征为主的医学数据特性,为随机森林设计参数随机搜索策略,
为KNN设计网格搜索策略,并结合动态参数约束机制防止过拟合,
最终基于交叉验证准确率选择最优参数组合。


\subsection{神经网络模型的构建}



\section{结果(Results)}



\section{讨论(Discussion)}









\section{引用}
\subsection{图片引用}
图片引用:如图 \ref{fig:enc-dec} 所示,图中。。。\textbf{引用对应的label,注意图片的路径即可。}
\textbf{latex会根据页面的篇幅自动调整图片的位置}。
如果不想浮动的图片或表格越过某些部分,在该部分前加\textbf{$\backslash$FloatBarrier}表明该部分不想被越过。
\begin{figure}[ht]
  \centering
  \includegraphics[scale=1.2]{figs/pic.png}
  \caption{Encoder-decoder结构}
  \label{fig:enc-dec}
\end{figure}



  文献引用:正如Hinton在文\cite{lecun2015deep}\cite{xu2015show}中所述。。。
\textbf{注意,文献的引用需要新建 filename.bib 文件,将文献对应的bitex格式的引用复制粘贴到文件中然后再引用!}
谷歌学术或微软学术均可生成文献引用可用的Bibtex格式。

\section{其他}
\subsection{公式}
公式等号对齐:
\begin{align}
  f_{att}(c_i, h_i) &= \boldsymbol{v_a}tanh(\boldsymbol{W_ac_i},\boldsymbol{W_bh_i})\\
  \alpha_i &= f_{att}(\boldsymbol{c_i}, \boldsymbol{h_i})\\
  \boldsymbol{s} &= softmax(\boldsymbol{\alpha})\\
  \boldsymbol{z} &= \sum{s_i\boldsymbol{c_i}}  
\end{align}

  其中 $\boldsymbol{c_i}$表示。。。

\subsection{代码与伪代码}
{\setmainfont{Courier New Bold} % 设置代码字体      
  % 代码段             
  \begin{lstlisting}
  #include <iostream>
  int main()
  {
      std::cout << "Hello, World!" << std::endl;
  }  
  \end{lstlisting}
                  
  \begin{lstlisting}
  #include <iostream>
  int main()
  {
      constexpr int MAX = 100;
  }  
  \end{lstlisting}
}

% 来自 https://blog.csdn.net/lwb102063/article/details/53046265
\makeatletter  
\def\BState{\State\hskip-\ALG@thistlm}  
\makeatother  
\begin{algorithm}  
\caption{My algorithm}\label{euclid}  
\begin{algorithmic}[1]  
\Procedure{MyProcedure}{}  
\State $\textit{stringlen} \gets \text{length of }\textit{string}$  
\State $i \gets \textit{patlen}$  
\BState \emph{top}:  
\If {$i > \textit{stringlen}$} \Return false  
\EndIf  
\State $j \gets \textit{patlen}$  
\BState \emph{loop}:  
\If {$\textit{string}(i) = \textit{path}(j)$}  
\State $j \gets j-1$.  
\State $i \gets i-1$.  
\State \textbf{goto} \emph{loop}.  
\State \textbf{close};  
\EndIf  
\State $i \gets i+\max(\textit{delta}_1(\textit{string}(i)),\textit{delta}_2(j))$.  
\State \textbf{goto} \emph{top}.  
\EndProcedure  
\end{algorithmic}  
\end{algorithm}  


\FloatBarrier % 在不想被浮动的表格/图片 越过的部分前加 \FloatBarrier

\bibliography{refs}
\bibliographystyle{plain}


\end{document}